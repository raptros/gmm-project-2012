\section{Next steps}
Getting a comparison of label propagation and Textgrounder for geolocation on
the uniform grid remains our primary first step.
In order to do this, we will need to recreate the uniform case of
\cite{rolleretal:12}, which means successfully getting Textgrounder up and
running on TACC.
Once we have a log file from the textgrounder run, we'll also be able to extract the cell information we
need to create labels for the label propagation system. We will also need to modify the existing wikipedia preprocessing scripts to have them output the link structure information. Once we have this information, we can set up label propagation so that both junto and textgrounder use the same training set of articles; textgrounder for building the language models, and junto for obtaining initial seed labels.
After that, we can run both the systems to get the predicted locations and supply these results to an evaluation mechanism.
Figuring out how to perform these steps on the simple case of uniform grid will allow us to
explore more sophisticated models, both for creating geographic cells (i.e.
using kd-trees), and for tuning the articles graph.
\par
Once we can resolve our current technical difficulties and reach this point, we
will be able to explore questions such as how similar are the locational
predictions produced by the IR model vs. the graph-based label-propagation
model, and how useful is the additional information obtained by running label propogation over non-geotagged articles.

The ultimate aim of this project is still to attempt to extend
geographic labelling to the rest of wikipedia. 
Once we figure out which, if any, label propagation models will work well
enough for this purpose, we can explore the possibilities made available in the
new information from the extended set.
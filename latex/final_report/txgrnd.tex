\subsection{Textgrounder}
So far, we have not been successful in using the Textgrounder system of 
\cite{wing-baldridge:11} and \cite{rolleretal:12} to accomplish our goals. 
Several issues stand in our way, including structural problems in the
Textgrounder codebase, mistakes and false starts in attempting to get
the system up and running, and various problems involving dataset and resource
availability.
\par
The corpora that Textgrounder operates on are usually large enough that fitting them
in memory is impossible; this requires Textgrounder to be able to
build its models while iterating over chunks of the corpora.
The current architecture of Textgrounder accomplishes this by using internal
iterators - requiring that everything that has to be done to a document be
coupled to the corpus-loading system.
\par
Unfortunately, this makes it difficult for us to use Textgrounder to achieve
our objective, which is to find the grid cell that a particular geotagged document is located in.
Our proposed label-propagation model uses the same principle (of dividing the
world into grid cells) as Textgrounder; in order to build our seed set, we need to
know the correct cell assignment (under a particular model for generating
cells) for every geo-tagged document.
The internal iterator structure makes it very difficult for us to use the
existing cell grid generation code, as we would most likely need to implement our own
loading harness.
\par
Benjamin Wing (who explained this issue) mentioned in an email conversation that he intends to re-design
Textgrounder to instead use external iteration; this would allow the cell-grid
system to be less tightly coupled to the loading system.
Such a re-design would make it much easier to build upon the parts of the
Textgrounder codebase that we need. However, our time constraints mean that we cannot wait for the new architecture to be implemented, nor will we have
time to aid the process by volunteering.
Instead, we will use a workaround suggested by Stephen Roller; we have built a
simple tool that parses Textgrounder logfiles, which contain listings of grid
cells with their geographic boundaries.
Admittedly, this is a fragile and inelegant solution, but it will allow us to
get the sets of labels we need for propagation as soon as we manage to run
Textgrounder itself.
\par
We have to admit that most of our lack of progress is the result of our own
poor decision-making and not understanding the systems we work with.
The key mistake we made was not focusing on getting Textgrounder up and running
on TACC as soon as possible; failing to do that led us to
attempt trying to run it on a personal computer.
It turns out that running Textgrounder's Wikipedia preprocessing scripts is
impractically slow on a personal machine. While it may be possible to actually
perform an evaluation (such as replicating the results of \cite{rolleretal:12})
on such a machine, we recommend against trying.
In the end, hours of work were wasted on figuring out how to set up
Textgrounder's environment correctly locally (and often having to repeat time consuming
steps after false starts), only for us to decide it was not worth the trouble.
\par
At this point, we decided to try to run Textgrounder on the Linguistics
department machines, where we eventually discovered that (contrary to the notes
in Textgrounder's README) the Wikipedia corpus is not correctly set up
there for running an evaluation.
Since we do not know what exactly is wrong with the setup on department machines, even if we had write
access to the relevant directories, we would not be able to fix the problem
ourselves.
As it turns out, we have been told that at least one other person
has been able to successfully run Textgrounder on the TACC cluster, so our plan
now is to do the same and perform the setup, preprocessing and evaluations there.

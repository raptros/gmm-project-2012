%!TEX root = final_report.tex
\subsection{Label propagation} 

In order to run label propagation, we first need to create a graph structure
from Wikipedia articles. In our project, this graph structure is meant to be
derived from the links between articles. Specifically, the graph is to be
represented using the adjacency list notation (v -$>$ link1 link2 ...), where
each vertex is a Wikipedia article and is associated with a list of vertices,
which are the other Wikipedia articles that the article has outgoing links to.

\par However, to the best of our knowledge, such a graph
representation cannot be directly obtained from the current implementation of
the Textgrounder framework, and will require additions to the preprocessing
code. There is some code in place that handles the links in an article, but the
extent to which this code can be reused or updated remains to be determined.

There are a few open source projects
\footnote{http://haselgrove.id.au/wikipedia.htm}
\footnote{https://github.com/mirkonasato/graphipedia}
\footnote{http://code.google.com/p/wikipedia-netflix/wiki/WikipediaNetflix} that
attempt to create this graph by parsing xml dumps of wikipedia pages. However,
each of these projects has its idiosyncrasies that makes interoperability with
Textgrounder difficult.  Most importantly, they all use a different scheme for
generating unique article IDs than the one used in Textgrounder. For some
reason, perhaps for ease of implementation, these tools use the fact that all
article titles on wikipedia are unique to extract the title for each article.
They do this by sorting the article titles lexicographically (or in the order of
input), and using the index of the article title in this sorted list as the
article ID. Reconciling the article IDs between the output of these tools and
Textgrounder is a nontrivial problem. Furthermore, due to different
preprocessing strategies (for dealing with disambiguation pages, Wikipedia
specific pages, external links, etc.), each of these tools generates a slightly
different set of articles than the one produced by Textgrounder. 

\par Following a conversation with Mike Speriosu and Ben Wing, we have decided
to invest some time upfront in updating the Textgrounder preprocessing script
to generate this graph for us. This is an unexpected source of delay, but we
must have a link graph before we can do any label propagation with the Junto
toolkit.

\par An alternative, that works around this issue, is to use a fully connected
graph with an edge between all pairs of articles as the input for label
propagation. This is similar to the approach used in the original label
propagation algorithm suggested in \cite{zhu2002learning}. In this scenario,
the performance of label propagation would depend only on the edge weights,
and not on the structure of the graph. However in this case, it's not clear to
us what the edge weights for our geolocation task should be. We believe that
the usual document similarity metrics based on word distributions are not
appropriate as edge weights, since they do not directly reflect the notion of
geographical similarity. For example, we suspect that document similarity
metrics alone would not be able to capture the fact that Austin and Houston
are more geographically similar than Austin and San Francisco. We are
interested in feedback on this hypothesis, especially on whether it is worth
spending some time testing it, as opposed to proceeding with modifications to
the Textgrounder preprocessing scripts.

%!TEX root = final_report.tex
\section{Introduction}

\comment{You can now enclose text in a comment block to comment it out inline}
Geographic information is relevant in several contexts; for example,
geographic information retrieval for exploring document collections in
\newcite{ding2000computing}, toponym resolution in historical texts by
\newcite{smith2001disambiguating}, summarizing travelogues and travel
recommendation in \newcite{hao-et-al:10}, socio-linguistic studies in \newcite
{eisenstein-smith-xing:11}  and targeted advertising on the Internet.  With
the widespread use of mobile devices, geographic information is becoming
increasingly ubiquitous and important.  In this project we propose a method
for predicting the geographic location of Wikipedia articles with a graph
based semi- supervised algorithm and a small amount of labeled data. Wikipedia
is a rich source of geographic information with a number of geotagged article
and corresponding natural language text that talks about those geographic
locations, and hence is a good source for building IR based geographic
language models.

\par Prior work on geolocating Wikipedia articles, e.g., \newcite{wing-baldridge:11}
 and  \newcite{rolleretal:12}, has so far only used a subset of
the English language Wikipedia for training; specifically, it uses those
articles that already have geolocation tags associated with them.  These
articles constitute a relatively small portion (about 5 \%) of the English language
Wikipedia. This is not surprising, since most
articles on Wikipedia are not about geographic locations, or objects at
specific locations. On the other hand, the number of articles that have links
to or from such geolocated articles is much larger, which can be a potential
source of useful information. \comment{find actual figures}

\par Earlier approaches used the content of the geotagged articles by building
unigram language models from the text of the articles during training, and
then calculating the similarity of test documents with these language models.
This ignores the potentially useful information present in the hyperlinks
between articles, which is a strong indicator of semantic and geographic
relatedness, and hence can be used to improve prediction accuracy.

\par Another assumption inherent in prior work is that each article is
associated with a single location, i.e. the location specified by the
coordinates for the article. While this is a fair assumption for some
articles, e.g., those about historical monuments and physical objects, this
assumption is not necessarily valid for articles about geographically
distributed concepts such as states (e.g., Texas) or personal biographies
(e.g., George Washington).

\par Our hypothesis is that we can infer the location for non-geotagged
articles from their incoming and outgoing links. Unlike prior works that take
a language modeling approach to geolocation, our proposed approach makes use
of the link structure of the articles. We use the label propagation algorithm
to infer a probabilistic distribution over locations for each non-geotagged
article, starting from a small seed set of geotagged articles. This should
allow the inclusion of the text from these new geo-tagged articles in the
language models for the locations they are strongly associated with, hence
improving the accuracy of the language modeling approaches.



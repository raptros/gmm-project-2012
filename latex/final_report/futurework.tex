\section{Future Work}  

As the results of our preliminary investigations are encouraging, we would
like to explore an number of additional avenues for improving the results of
our approach. In the rest of this section we discuss the potential sources of
improvements and future work.

\par Firstly, there are a few options to consider while determining the
structure of the graph. The first question is whether the graph should be
directed, with edges pointing from referring document to the referred
document, or undirected.

\par The second set of questions pertain to the method to be used for inducing
the edges between the documents which serve as the vertices in the graph. Some
of these questions are :

\begin{enumerate}
    \item Should the edges be based solely on hyperlinks between nodes, or also based on similarity of their texts?
    \item Should the edges be transitive?
    \item Should we induce a dense graph with edges between all nodes or a sparse graph with a few edges with high weights? The first approach is similar to the one used in the original label propagation algorithm suggested in \newcite{zhu2002learning}, 
\end{enumerate}

Additionally, we could induce an edge between all documents that fall in a
particular grid cell to increase locality, or between documents in adjacent
cells, to introduce smoothing.

\par A third challenge is determining the weights of edges. Intuitively, the
weight of an edge should be proportional to the similarity of the two nodes
(documents). This notion of similarity can be formalized in a number of ways.
One approach would be to take the similarity between the language models for
the two documents as the edge weights. This could be the cosine similarity,
tf-idf weighted cosine similarity, or the KL divergence score for the document
pair.  In our experiments, we use the straightforward alternative of using the
same edge weight for all edges regardless of document similarity.

 
\par Furthermore, since each document can potentially be related to more than
one location, we can examine the performance of holding seed document label
distributions fixed versus allowing them to vary (and acquire non-zero weights
for other locations). The latter approach would allow a document label to be
influenced by related geographic locations;  for example, the distribution for
the article for Lake Austin would end up having a nonzero weight for the
location for Travis County. As suggested earlier, this might be useful in
geolocating concepts that span multiple locations.


\par Another issue that we didn't explore due to time constraints is finding
the optimal parameters (for the label propagation algorithm). While the label
propagation algorithm converges \comment{yes?}

\par We would also like to explore the label propagation results for a larger
graph consisting of non-geotagged documents using Hadoop. Apart from
implementation issues, which are hopefully already dealt with in the Hadoop
based implementation of Junto, a big challenge is evaluating the model
performance. This is problematic since we don't have the true grid cell labels
for the non-geotagged articles, if the articles can be assigned a geographic
location at all.

\par Also, we want to qualitatively explore the results generated by label
propagation, especially the plausibility \comment{not feasibility, what's the
word? - fixed: plausibility} of the distribution of labels for each test node.
Specifically, we want to qualitatively confirm if the label propagation
algorithm generates a sensible distribution of locations for a document even
when it doesn't get the exact location right. This property that we hypothesize
to hold is critical for the feasibility of integrating label propagation with
Textgrounder.

\par Finally, if label propagation
gives promising results, we can then examine the possibility of augmenting the
pseudo- documents in language-models based approaches with the additional
information from the new documents geolocated using label propagation.


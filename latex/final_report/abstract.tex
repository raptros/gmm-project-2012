\begin{abstract} 
We believe that the link structure of Wikipedia contains
information that would be useful for geolocation, but hasn't been used so far.
We propose that by using label-propagation algorithms to push geolocation
information around that link structure, we will be able to explore new
possibilities for geolocation, especially locating non-geotagged articles.
Furthermore, the labeling model gives us a way to associate articles with
multiple locations, which would allow bringing in new information about closely-
associated locations. The results of this work could be used  to aid the
traditional IR based geolocation models by providing them  with additional data
to learn their language models from. 

\par Unfortunately, our progress has been stymied by a host of technical
issues; these issues directly affect our ability to obtain label information
and graph structures. We have been working on overcoming these issues so that
we can construct and evaluate our models. 
\end{abstract}

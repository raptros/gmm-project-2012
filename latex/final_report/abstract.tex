\begin{abstract}  

We show that the link structure of Wikipedia contains information that hasn't
been used so far but is useful for geolocation. We show that by using label-
propagation algorithms to push geolocation information around the Wikipedia
link graph, we can explore new possibilities for geolocation, especially
locating non-geotagged articles. Furthermore, the labeling model gives us a
way to associate articles with multiple locations, which would allow bringing
in new information about closely- associated locations. We believe that the
results of this work can be used to aid the traditional IR based geolocation
models by providing them  with additional data to learn their language models
from.

\end{abstract}

\section{Next steps} Getting a comparison of label propagation and Textgrounder
for geolocation on the uniform grid remains our primary first step. 

\par In order to do this, we will need to recreate the uniform case of
\cite{rolleretal:12}, which means successfully getting Textgrounder up and
running on TACC. Once we have a log file from the Textgrounder run, we will also
be able to extract the cell information we need to create labels for the label
propagation system.  

\par  We will also need to modify Textgrounder's existing Wikipedia
preprocessing scripts to have them output the link structure information. Once
we have this information, we can set up label propagation so that both Junto
and Textgrounder use the same training set of articles; Textgrounder for
building the language models, and Junto for obtaining initial seed labels.

\par Once Junto and Textgrounder are set up, we can run both the systems to
get the predicted locations and supply these results to an evaluation
mechanism. Figuring out how to perform these steps on the simple case of
uniform grid will allow us to explore more sophisticated models, both for
creating geographic cells (i.e. using kd-trees), and for experimenting with
the article graph. 

\par Once we can resolve our current technical difficulties
and reach this point, we will be able to explore questions such as how similar
are the locational predictions produced by the IR model vs. the graph-based
label- propagation model, and how useful is the additional information
obtained by running label propagation over non-geotagged articles. 

\par To evaluate the performance of the models, we will use measures based on
error distances, e.g., mean and median error between predicted and actual
distance, instead of IR based methods like precision or recall.
\cite{rolleretal:12} show that using the centroid of locations of documents in
the grid cell as the cell's location results in more accurate predictions than
using the mean of document locations (as in \cite{wing-baldridge:11}), even in
the uniform-grid case.

\par The ultimate aim of this project is to get a reasonable geolocation
accuracy on non- geotagged articles using label propagation, so that these new
articles can be included in the language model for Textgrounder cells. Once we
figure out which, if any, label propagation models will work well enough for
this purpose, we can explore the possibilities made available in the new
information from the extended set.

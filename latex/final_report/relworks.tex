\section{Related Work} 

\subsection{Geolocation}   

Traditional geolocation systems have used either
gazetteer location information, or an information retrieval system, as
explained in \cite{skiles:12}. Most IR-based approaches divide the world into
discrete grid cells, each of which are treated as pseudo-documents that
consist of texts from the documents located in them. New documents are then
geolocated by comparing their language model with the language models of each
of the grid cells and assigning the documents to the cells with which they
have the highest similarity scores. \cite{eisenstein-smith-xing:11} show the
effectiveness of this language modeling approach by explicitly removing geo-
references from documents and predicting their location using only the
linguistic and dialectical features present in their text.

\par The use of Wikipedia for article geolocation, classification and toponym
resolution was suggested in \cite{overell2009geographic}. While they only use
the metadata associated with the articles, others have extended that work by
adopting the language modeling approach to geolocation.  This is the approach
used by \cite{wing-baldridge:11}, who divide the world into a uniform grid of
equal cell size. \cite{rolleretal:12} developed an adaptive grid based on KD-
trees that splits grid cells such that they contain roughly equal number of
documents each. A different model for geolocation is presented by
\cite{eisensteinetal:11}; they use a general generative model as an
alternative to LDA, and model distributions as a mixture of gaussians over the
earth's surface instead of discrete grid cells.

\par Furthermore, \cite{kumar-et-al:11} show that language models learned from
Wikipedia can be used for prediction tasks in other unrelated domains by
predicting publication date for books from Project Gutenberg. This is a
promising result that suggests that the language models learned from Wikipedia
are generalizable to other domains and may be used for geolocating documents
from other sources.

\subsection{Label Propagation} 

Label propagation is a general purpose graph- based semi-supervised learning
algorithm, as described in \cite{zhu2002learning} and \cite{talukdar:09}.
Given a graph G = \{V, E, W\}, where V are the graph vertices, E are the edges
and W is a matrix representing the edge weights, the label propagation
algorithm produces a set of class labels for each node in the graph starting
from a small seed set of labeled nodes. The algorithm iteratively propagates
the class labels on the seed nodes to their neighbors in proportion to the
weight of the edge between them. The edge weights are chosen to be
proportional to the similarity between the nodes.
\comment{write about MAD.}

\par Label propagation has been successfully used for sentiment analysis
\cite{speriosu2011twitter}, community detection \cite{raghavan2007near},
classification and ranking \cite{talukdar:10}, and video recommendations on
youtube \cite{baluja2008video}.

\section{Results} 

As an indicator of the performance of our model, we report the learning curves
for the label propagation algorithm.  To measure the performance of the model,
we calculate the mean error distance, median error distance, precision and
recall at k curves for different amounts of seed data. In our model, the error
distance is defined as the distance between the centers of the predicted and
the actual grid cells. \comment{confirm if the centers of the grid cells are
centroids or geometric centers} With this definition, the mean error distance
is the mean of the error distances for all the test documents. Similarily, the
median error distance is the median of errors for all the documents.
\comment{We also show the error curves, which are a better indicator of system
performance.} Precision is defined as the fraction of test documents for which
the model predicts the correct grid cell. Since our model returns a ranked
list of probable grid cells for each document, \comment{more motivation for
why this is a good thing} we also calculate the recall at k curves for the
results. For our experiments, rekall at k is defined as the fraction of the
test documents for which the correct label is present within to top k grid
cells predicted by label propagation.

It is to be noted that the error distances are sensitive to the grid size,
since they are calculated as the distance between the centers of predicted and
the actual cells. Because of this, a smaller grid size (up to a point) results
in better mean error distances since the penalty for predicting a neighboring
but only partly correct cell is lower. We leave repeating the experiments with
different grid sizes and adaptive grids as future work.
